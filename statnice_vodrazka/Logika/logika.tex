\documentclass[a4paper]{article}      % Comments after  % are ignored
\usepackage{amsmath,amssymb,amsfonts} % Typical maths resource packages
\usepackage[utf8]{inputenc}
\usepackage{czech}
\usepackage{graphicx}
\usepackage{latexsym}

\newtheorem{theorem}{Věta}[section]
\newtheorem{lemma}[theorem]{Lemma}
\newtheorem{proposition}[theorem]{Předpoklad}
\newtheorem{corollary}[theorem]{Důsledek}

\newenvironment{proof}[1][Důkaz]{\begin{trivlist}
\item[\hskip \labelsep {\bfseries #1}]}{\end{trivlist}}
\newenvironment{definition}[1][Definice]{\begin{trivlist}
\item[\hskip \labelsep {\bfseries #1}]}{\end{trivlist}}
\newenvironment{example}[1][Příklad]{\begin{trivlist}
\item[\hskip \labelsep {\bfseries #1}]}{\end{trivlist}}
\newenvironment{remark}[1][Pozorování]{\begin{trivlist}
\item[\hskip \labelsep {\bfseries #1}]}{\end{trivlist}}

\newcommand{\qed}{$\blacksquare$}

\begin{document}

\section*{Axiomy, pravidla a věty}

\subsection*{\textsl{Axiomy:}}
\begin{description}
	\item[A1]$ A \rightarrow (B\rightarrow A)$
	\item[A2]$ (A\rightarrow (B \rightarrow C)\rightarrow((A\rightarrow B)\rightarrow (A\rightarrow C)))$
	\item[A3]$ (\neg B\rightarrow \neg A)\rightarrow(A\rightarrow B)$
	\item[A4]$ (\forall x)A \rightarrow A_{x}[t]$ \\ tzv. \emph{schema specifikace}
	\item[A5]$ (\forall x)(A \rightarrow B)\rightarrow(A\rightarrow (\forall x)B)$ - pokud $A$ neobsahuje volné $x$\\
	tzv. \emph{schema přeskoku} 
\end{description}

\subsection*{\textsl{Pravidla:}}
\begin{description}
	\item[Modus Ponens:] $A,(A \rightarrow B) \vdash B$
	\item[Veta o Dedukci:] $T,A \vdash B$ právě když $T \vdash A \rightarrow B$
	\item[Generalizace:] $A \vdash (\forall x) A$
	\item[Lemma pro $\exists$:] $\vdash A_{x}[t] \rightarrow (\exists x) A$
	\item[Zavedení $\forall$:] pokud $\vdash A \rightarrow B$ a $x$ není volná v $A$\\ 
	potom platí: $\vdash A \rightarrow (\forall x) B$
	\item[Zavedení $\exists$:] pokud $\vdash A \rightarrow B$ a $x$ není volná v $B$\\ 
	potom platí: $\vdash (\exists x) A \rightarrow B$ 
\end{description}

\subsection*{\textsl{Věty:}}
\begin{description}
	\item[V1]	$A\rightarrow A$
	\item[V2]	$\neg A \rightarrow (A \rightarrow B)$
	\item[V3]	$\neg \neg A \rightarrow A$
	\item[V4]	$A\rightarrow \neg \neg A$
	\item[V5]	$(A \rightarrow B) \rightarrow (\neg B \rightarrow \neg A)$
	\item[V6]	$A \rightarrow (\neg B \rightarrow \neg (A \rightarrow B))$
	\item[V7]	$(\neg A \rightarrow A) \rightarrow A$
	\item[V8]	$A \wedge B \vdash A$\\
						$A \wedge B \vdash B$
	\item[V9]	$A,B \vdash A \wedge B$ 
\end{description}

\section{První část}

\subsection{Formální systém} sestává ze tří částí:
\begin{enumerate}
\item \textbf{jazyk} (symboly ze kterých vytváříme konečné posloupnosti - termy, formule)
\item \textbf{axiomy} (formule přijaté jako základní tvrzení)
\item \textbf{odvozovací pravidla} (syntaktická pravidla pro odvozování nových formulí)
\end{enumerate}

\paragraph{Jazyk 1. řádu} je společný pro všechny formální systémy 1. řádu (tzn. pracují pouze s \emph{proměnnými jednoho typu}).
Tento jazyk obsahuje:
\begin{itemize}
	\item symboly pro \emph{proměnné}
	\item \emph{logické spojky} $\wedge,\ \vee,\ \rightarrow,\ \leftrightarrow,\ \neg$
	\item \emph{kvantifikátory} $\forall,\ \exists$
	\item symboly pro \emph{predikáty}
	\item symboly pro \emph{funkce}
	\item pomocné symboly $(,\ ),\ \lbrace,\ \rbrace,\ [,\ ],\ldots$
\end{itemize}

\paragraph{Termy vs. formule} - \emph{termy} popisují objekty vzniklé pomocí nějakých (v termu popsaných) operací:
\[
f(g(x))
\]
\emph{formule} vyjadřují nějaké tvrzení (které je buď pravdivé nebo nepravdivé):
\[
(\forall x)  (x > 0)
\]

\begin{example}
\emph{Formální systém výrokové logiky}:\\
\textbf{P} - množina prvotních formulí.
\begin{enumerate}
\item Jazyk: logické spojky, symbly pro prvotní formule z \textbf{P} a závorky
\item Axiomy: A1, A2, A3
\item Odvozovací pravidlo: Modus ponens - z formulí $A$ a $A \rightarrow B$ odvoď formuli $B$  
\end{enumerate}
\end{example}

\begin{example}
\emph{Formální systém predikátové logiky 1. řádu:}
\begin{enumerate}
\item Jazyk: jazyk 1. řádu
\item Axiomy: 	\begin{itemize}
			\item axiomy pro logické spojky - A1, A2, A3
			\item axiomy pro kvantifikátory - Schema specifikace, Schema přeskoku
		\end{itemize}
\item Odvozovací pravidla: Modus ponens, Pravidlo generalizace - pro libovolnou proměnnou $x$ z formule $A$ odvoď formuli $(\forall x)A$
\end{enumerate}
\end{example}

\paragraph{Důkaz ve formálním systému} je konečná posloupnost formulí ve které je každá formule axiomem, nebo je odvozená z předchozích formulí pomocí odvozovacích pravidel. Formule, ke které máme důkaz je \emph{dokazatelná}.

\section{Druhá část}

\subsection{Výroková logika} zkoumá syntax a sémantiku formulí vytvořených pouze pomocí logických spojek.
Všechny složitější formule bereme jako prvotní formule - výroky, které buď platí, nebo ne.\\
Jazyk navíc můžeme redukovat - pokud jako základní logické spojky vezmeme $\neg$ a $\rightarrow$, můžeme pomocí nich vyjádřit všechny ostatní.

\paragraph{Sémantika výrokové logiky} - abychom mohli rozhodovat o pravdivosti formulí ve výrokové logice, potřebujeme někde zjistit
jak je to s pravdivostí prvotních formulí ($p \in \mathbf{P}$, které v rámci VL nepitváme).
Pro tento účel se zavádí tzv. \emph{pravdivostní ohodnocení}:
\[
v: \mathbf{P} \rightarrow \lbrace 0,1 \rbrace
\]
Indukcí podle složitosti formule lze potom určit pravdivost libovolné formule ($\mathbf{F}$ - množina všech formulí vytvořených z prvotních flí. pomocí logických spojek) pomocí rozšířeného zobrazení:
$\bar{v}: \mathbf{F} \rightarrow \lbrace 0,1 \rbrace$

\paragraph{Tautologie a splnitelnost}

\begin{definition}[Tautologie] je formule $A \in \mathbf{F}$, která je pravdivá při libovolném pravdivostním ohodnocení $v$. To znamená, že pro libovolnou volbu $v$:
\[
\bar{v}(A) = 1
\]
(Pozn. $\bar{v}$ je \emph{rozšířené} pravdivostní ohodnocení $v$)
\end{definition}

\begin{definition}[Splnitelná formule] je každá, formule $A \in \mathbf{F}$, pro kterou existuje alespoň jedno pravdivostní ohodnocení $v$
při kterém $\bar{v}(A) = 1$.
\end{definition}

\begin{definition}[Model množiny formulí T] je každé ohodnocení $v$ při kterém jsou splněny všechny formule z T.
\end{definition}

\begin{definition}[Tautologický důsledek množiny formulí T] je formule $A$, která je pravdivá při každém ohodnocení, které je modelem $T$.
Značení: $T \models A$ znamená, že $A$ je tautologickým důsledkem množiny $T$.
\end{definition}

\begin{definition}[Dokazatelnost]
Konečná posloupnost formulí $A_{1},\ldots,A_{n}$ je \emph{důkazem} formule $A$, jestliže $A=A_{n}$ a pro každé $i$ takové, že $1 \leq i \leq n$
je $A_{i}$ buď axiom, nebo je odvozena z předchozích formulí pomocí pravidla modus ponens. Pokud existuje důkaz formule $A$, říkáme, že $A$ je dokazatelná ve výrokové logice (je větou výrokové logiky) a píšeme $\vdash A$.
Dokazatelnost $A$ z předpokladů $T$ značíme $T \vdash A$. Přitom formule z množiny $T$ se v důkazu používají jako jakési "přidané axiomy".  
\end{definition}

\begin{theorem}[Věta o dedukci]
\[
T \vdash A \rightarrow B\ \Leftrightarrow\ T \cup \lbrace A \rbrace \vdash B
\]
\end{theorem}

\begin{proof}
\begin{itemize}
\item "$\Rightarrow$": Protože platí $T \vdash A \rightarrow B$, musí existovat posloupnost formulí:
\[
A_{1},\ldots,A_{n},A \rightarrow B
\]
přidáme-li $A$ do předpokladů, bude možné přidat $A$ na počátek této posloupnosti a odvodit $B$ pomocí pravidla modus ponens.
Důkaz $B$ z $T \cup \lbrace A \rbrace$ bude tedy:
\[
A,A_{1},\ldots,A_{n},A \rightarrow B,B
\]
takže $T \cup \lbrace A \rbrace \vdash B$.
\item "$\Leftarrow$": Mějme posloupnost $A = A_{0},A_{1},\ldots,A_{n} = B$, která je důkazem pro $T \cup \lbrace A \rbrace \vdash B$.
Indukcí podle $i$ dokážeme:
\[
T \vdash A \rightarrow A_{i}
\]
V každém indukčním kroku vždy můžou pro $A_{i}$ nastat dva případy:
\begin{enumerate}
	\item $A_{i}$ je axiom, nebo patří do $T \cup \lbrace A \rbrace$
	\item $A_{i}$ je odvozena pomocí modus ponens z nějakých formulí $A_{j}$ a $A_{k}$ takových, že $j,k < i$
\end{enumerate}
\end{itemize}
\end{proof}

\begin{theorem}[Věta o kompaktnosti]
Množina formulí $T$ je splnitelná, právě když je splnitelná libovolná konečná podmnožina $T_{0} \subseteq T$.
\end{theorem}
\begin{proof}
\begin{itemize}
\item "$\Rightarrow$" - $T$ je splnitelná, takže má model $v$. V tomto modelu je potom splněna libovolná podmnožina $T$.
\item "$\Leftarrow$" - pokud je $T$ nejvýše spočetná (konečná, nebo nekonečná, ale každopádně spočetná) tak
lze dokázat indukcí dle velikosti podmnožiny. Jinak se to dokazuje přes větu o kompaktnosti součinu kompaktních topologických prostorů.
\end{itemize}
\end{proof}

\begin{lemma}[Lemma o prvotních formulích]\label{lemma:prvotni_formule}
Značení: pokud $v$ je ohodnocení a $B$ je formule, potom
\[
\bar{v}(B) = \left\{
\begin{array}{lcl}
1 & \Rightarrow & B^{v} = B\\
0 & \Rightarrow & B^{v} = \neg B\\
\end{array}
\right.
\]
Znění lemmatu: Nechť $A$ je formule VL a $P_{1},\ldots,P_{n}$ jsou všechny její prvotní podformule. Potom
platí:
\[
P_{1}^{v},\ldots,P_{n}^{v} \vdash A^{v}
\]
\end{lemma}

\begin{lemma}[Lemma o neutrální formuli]\label{lemma:neutralni_formule}
Pro množinu formulí $T$ a formule $A$ a $B$ platí:
\[
(T,A \vdash B)\ \wedge\ (T,\neg A \vdash B)\ \Leftrightarrow\ (T \vdash B)
\]
\end{lemma}

\begin{theorem}[Slabá věta o úplnosti (Post)]
Ve výrokové logice jsou dokazatelné právě tautologie.
\[
\vdash A\ \Leftrightarrow\ \models A
\]
\end{theorem}

\begin{proof}
Idea důkazu.
\begin{itemize}
	\item "$\Rightarrow$": dokážeme, že všechny axiomy VL jsou tautologiemi a že pravidlo modus ponens dělá z tautologií zase tautologie.
	\item "$\Leftarrow$": Vezmeme libovolnou formuli $A$ a pracujeme s jejími prvotními podformulemi $P_{1},\ldots,P_{n}$.
Potřebujeme lemmata \ref{lemma:prvotni_formule} a \ref{lemma:neutralni_formule}. Protože je $A$ tautologie, můžeme zvolit ohodnocení $v$
takové, že při něm budou všechny prvotní formule pravdivé. Podle lemmatu \ref{lemma:prvotni_formule} dostaneme:
\[
P_{1},\ldots,P_{n} \vdash A
\]
Můžeme dále zvolit takové ohodnocení, že všechny prvotní formule jsou pravdivé až na $P_{n}$, tedy:
\[
P_{1},\ldots,\neg P_{n} \vdash A
\]
Použitím lemmatu \ref{lemma:neutralni_formule} na předchozí dva výsledky dostaneme:
\[
P_{1},\ldots,P_{n-1} \vdash A
\]
Stejný postup aplikujeme dokud nezískáme $\vdash A$.
\end{itemize}
\end{proof}

\begin{theorem}[Věta o úplnosti Výrokové logiky]
Nechť $T$ je množina formulí a $A$ je formule. Potom platí:
\[
(T \vdash A)\ \Leftrightarrow\ (T \models A)
\]
\end{theorem}


\paragraph{Tvary formulí (konjunktivně disjunktivní a disjunktivně konjunktivní)}. Každou formuli výrokové logiky můžeme vyjádřit ve standartním tvaru. Zavedeme nejprve pojmy:
\begin{itemize}
\item \emph{literál} - prvotní formule, nebo její negace
\item \emph{klauzule} - disjunkce literálů
\end{itemize}

\paragraph{Konjunktivní tvar.} Formule je v konjunktivním tvaru, pokud je zapsána jako konjunkce klauzulí. Př.:
\[
(\neg a\vee b\vee c)\wedge(a\vee \neg b\vee c)\wedge(a\vee b\vee \neg c)
\]

\paragraph{Disjunktivní tvar.} Formule je v disjunktivním tvaru, je-li zapsána jako disjunkce konjunkcí literálů. Př.:
\[
(x\wedge \neg y\wedge \neg z)\vee(\neg x\wedge y\wedge \neg z)\vee(\neg x\wedge \neg y\wedge z)
\]

\begin{theorem}[Standartní tvary formulí]
Ke každé formuli $A$ lze sestrojit formuli $A_{k}$ v konjunktivním tvaru a formuli $A_{d}$ v disjunktivním tvaru tak, že:
\[
(\vdash A \leftrightarrow A_{k})\ \wedge\ (\vdash A \leftrightarrow A_{d})
\]
\end{theorem}

\section{Třetí část}

\subsection{Predikátová logika} používá jazyk prvního řádu.
Oproti výrokové logice obsahuje tedy navíc ještě proměnné, kvantifikátory, predikáty a funkce.
Speciálním případem predikátu je predikát rovnosti.

\paragraph{Termy} jsou zavedeny induktivně:
\begin{enumerate}
\item každá proměnná je \emph{term} 
\item jsou-li $t_1,\ldots,t_{n}$ termy a  $f$ je n-ární funkční symbol, potom $f(t_1,\ldots,t_{n})$ je taky term.
\item Všechny výrazy vytvořené konečným použitím výše uvedených pravidel jsou termy.
\end{enumerate}

\paragraph{Formule} jsou konstruovány následovně:
\begin{enumerate}
\item Jsou-li $t_1,\ldots,t_{n}$ termy a $p$ je n-ární predikátový symbol, potom $p(t_1,\ldots,t_{n})$ je \emph{atomická formule}.
\item Jsou-li $A$ a $B$ formule, potom přidáním logických spojek lze vytvořit formule.
\item Je-li $A$ formule a $x$ proměnná, potom $(\forall x)A$ a $(\exists x)A$ jsou taky formule.
\item Cokoliv vznikne konečným použitím výše uvedených pravidel jsou formule.
\end{enumerate}

\paragraph{Vázaný a volný výskyt proměnné.} Proměnná $x$ je ve formuli $A$ \emph{vázaná}, pokud je součástí nějaké podformule ve tvaru
$(Q\ x)A$ kde $Q$ je jeden z kvantifikátorů. Pokud není proměnná nikde vázána, je to \emph{volná} proměnná.
Proměnná může mít v jedné formuli volný i vázaný výskyt.

\paragraph{Otevřená a uzavřená formule.} Formule $A$ je \emph{uzavřená}, pokud neobsahuje žádnou volnou proměnnou. Pokud formule $A$ neobsahuje žádnou vázanou proměnnou, je \emph{otevřená}.

\paragraph{Realizace jazyka $L$} obsahuje:
\begin{itemize}
\item neprázdnou množinu individuí $M$
\item zobrazení $f: M^{n} \rightarrow M$ pro každý n-ární funkční symbol
\item n-ární relaci $p \subseteq M^{n}$ pro každý n-ární predikátový symbol 
\end{itemize}
Konstanty chápeme jako nulární funkční symboly.

\paragraph{Realizace termů} je definována pomocí zobrazení $e$, které každé proměnné přiřadí prvek z $M$.
\begin{enumerate}
\item Je-li termem $t$ proměnná $x$, potom $t[e] = e(x)$
\item Je-li termem $t$ n-ární funkční symbol $f(t_1,\ldots,t_{n})$ potom $t[e] = f(t_1[e],\ldots,t_{n}[e])$ 
\end{enumerate}

\paragraph{Tarského definice pravdy.} Pokud je $\mathcal{M}$ realizace jazyka $L$ a $e$ je ohodnocení proměnných, pravdivost formule
$A$ v realizaci $\mathcal{M}$ při ohodnocení $e$ se značí:
\[
\mathcal{M} \models A[e]
\]
A je definována induktivně podle složitosti formule $A$.

\begin{definition}[Teorie prvního řádu]
Mějme jazyk $L$ prvního řádu a množinu formulí $T$ z $L$. Potom $T$ je teorie prvního řádu v jazyce $L$.
Formule z $T$ jsou jejími \emph{speciálními axiomy}. Speciálně pro predikátovou logiku 1. řádu platí: $T = \emptyset$.
\end{definition}

\paragraph{Dokazatelnost} v predikátové logice pouze využívá další axiomy a odvozovací pravidla.
Důkaz je stále tvořen posloupností formulí, z nichž každá je sama axiomem,nebo je odvozena podle pravidla modus ponens, nebo generalizace.

\begin{theorem}[Věta o uzávěru]
Mějme formuli $A$ a její uzávěr $A'$. Platí:
\[
(\vdash A)\ \Leftrightarrow\ (\vdash A')  
\]
Důkaz jedním směrem: pravidlo generalizace.\\
Důkaz druhým směrem: lemma $\vdash (\forall x_1)\ldots(\forall x_n)A \rightarrow A_{x_1,\ldots,x_{n}}[t_1,\ldots,t_{n}]$ a pravidlo modus ponens.
\end{theorem}

\paragraph{Věta o dedukci} se dokazuje podobně jakove výrokové logice - pouze musíme navíc počítat s odvozovacím pravidlem generalizace.
V předpokladech je navíc požadavek nauzavřenost formule $A$.\\
Důležitým důsledkem věty o dedukci je následující tvrzení:

\begin{corollary}[Důkaz sporem]
Ať $A'$ je uzávěr $A$ a $T$ je množina formulí. Potom:
$T \vdash A$ právě když $T \cup \lbrace \neg A'\rbrace$ je sporná teorie (pro libovolnou formuli $B$ v ní lze dokázat $B$ a zároveň $\neg B$).
\end{corollary}

\paragraph{Substituovatelnost.} Term $t$ je substituovatelný za proměnnou $x$ do formule $A$, pokud pro každou proměnnou $y$ v $t$, žádná kvantifikovaná podformule $A$ tvaru $(Q\ y) B$ neobsahuje volný výskyt $x$ (z hlediska $A$).
Pokud by existoval volný výskyt proměnné $x$,
mohli bychom vytvořit nevhodnou substitucí z neuzavřené formule uzavřenou a to nechceme. Značení: $A_{x}[t]$.
Úspěšnou substitucí vytvoříme \emph{instanci} dané formule.

\begin{example}[Špatná substituce]
Máme formuli $A = (\forall y) (x < y + 1)$ a substituujeme term $t = y + 5$ za proměnnou $x$:
\[
A_{x}[t] = (\forall y)(y + 5 < y + 1)
\]
Což se úplně nepovedlo. Aby k tomu nedošlo, zavádí se pojem \emph{substituovatelnosti}.
Substituovatelnost je zaručená, pokud žádná proměnná obsažená v $t$ není v $A$ vázaná.
\end{example}
 
\begin{theorem}[Věta o konstantách]
Mějme množinu formulí $T$ a formuli $A$ z jazyka $L$. Jsou-li $c_1,\ldots,c_{n}$ nové konstanty (rozšíříme $L$ na $L'$) a
$x_1,\ldots,x_{n}$ proměnné, potom platí:
\[
(T \vdash A_{x_1,\ldots,x_{n}}[c_1,\ldots,c_{n}])\ \Leftrightarrow\ (T \vdash A)
\]
\end{theorem}

\begin{proof}
\begin{itemize}
\item "$\Leftarrow$": je-li dokazatelná $A$, je dokazatelná i každá její instance.
\item "$\Rightarrow$": je-li dokazatelná $A_{x_1,\ldots,x_{n}}[c_1,\ldots,c_{n}]$, znamená to, že má důkaz. Všechny konstanty nahradíme
proměnnými $y_1,\ldots,y_{n}$ a důkaz který máme bude fungovat dál. 
\end{itemize}
\end{proof}

\paragraph{Prenexní tvary formulí.} Formule $A$ je v prenexním tvaru, má-li tvar:
\[
(Q_1\ x_1)(Q_2\ x_2)\ldots(Q_n\ x_n)B
\]
kde $Q_i$ je jeden z kvantifikátorů $\forall$ a $\exists$, $B$ je otevřená formule a $x_1,\ldots,x_n$ jsou navzájem různé proměnné.

\begin{theorem}[O prenexním tvaru formulí]
Ke každé formuli $A$ lze sestrojit formuli $A'$ v prenexním tvaru tak, že:
\[
\vdash A \leftrightarrow A'
\]
V důkazu se využívá nahrazení formule variantou (přejmenování proměnných) a důkaz 4 prenexních operací:
\begin{enumerate}
\item záměna kvantifikátoru
\[
\neg (\mathbf{Q}\ x)A \leftrightarrow (\mathbf{\bar{Q}}\ x)\neg A
\] 
\item kvantifikace implikace "proti šipce" (předpoklad $x$ není volná v $A$)
\[
(\mathbf{Q}\ x)(A \rightarrow B) \leftrightarrow (A \rightarrow (\mathbf{Q}\ x)B)
\]
\item kvantifikace implikace "přes závorku" (předpoklad $x$ není volná v $B$)
\[
(\mathbf{\bar{Q}}\ x)(A \rightarrow B) \leftrightarrow ((\mathbf{Q}\ x)A \rightarrow B)
\]
\item kvantifikace konjunkce/disjunkce ($\ \square \rightsquigarrow \wedge\ |\ \vee$, předpoklad $x$ není volná v $B$) 
\[
((\mathbf{Q}\ x)A \square B) \rightarrow (\mathbf{Q}\ x)(A \square B)
\]
\end{enumerate}
\end{theorem}

\section{Čtvrtá část}

\begin{theorem}[Věta o korektnosti]
Ať $T$ je teorie s jazykem $L$. Pokud pro nějakou formuli $A$ z $L$ platí:
\[
T \vdash A
\]
potom $T \models A$. Takže každá věta teorie $T$ je splněna ve všech modelech $T$.
\end{theorem}

\begin{proof}
Věta o korektnosti se dokazuje indukcí podle délky důkazu formule.
Ať je $\mathcal{M}$ libovolný model teorie $T$ a $A_1,\ldots,A_n = A$ je důkaz $A$ z $T$.
Indukcí dokážeme, že $\forall i:\ \mathcal{M} \vDash A_i$. Rozbor příkladů z čeho vzešlo $A_i$:
\begin{enumerate}
\item je to formule z $T$ - takže je splněna v $\mathcal{M}$
\item je to axiom PL:
	\begin{enumerate}
	\item je to axiom VL
	\item je to axiom \emph{specifikace}
	\item je to axiom \emph{přeskoku}
	\item je to axiom rovnosti
	\end{enumerate}
\item je odvozeno pravidlem \emph{modus ponens}
\item je odvozeno pravidlem \emph{generalizace}
\end{enumerate}
\end{proof}

\begin{definition}[Úplná teorie]
Teorie $T$ s jazykem  $L$ je \emph{úplná}, jestliže $T$ je bezesporná
a pro každou \emph{uzavřenou} formuli $A$ jazyka $L$ je jedna z formulí $A$,$\neg A$ v $T$ dokazatelná.
\end{definition}

\begin{definition}[Henkinova teorie]
Teorie $T$ s jazykem $L$ je \emph{Henkinova} pokud pro každou uzavřenou formuli $(\exists x) B$ existuje v $L$ konstanta $c$ tak,
že:
\[
T \vdash (\exists x)B \rightarrow B_{x}[c]
\]
\end{definition}

\begin{lemma}[Model Henkinovy teorie]
Je-li $T$ úplná Henkinova teorie, pak má model.
\end{lemma}

\begin{theorem}[Věty o úplnosti (G\"{o}del)]
Ať $T$ je teorie s jazykem $L$. Potom platí:
\begin{enumerate}
\item je-li $A$ formule z $L$, potom:
\[
(T \vdash A)\ \Leftrightarrow\ (T \models A)
\]
\item $T$ je bezesporná právě když má nějaký model.
\end{enumerate}
\end{theorem}

\begin{remark}
Z 2. části věty o úplnosti vyplývá 1. část:
\begin{itemize}
\item Mějme teorii $T$ a formuli $A$ - obojí v jazyce $L$.
\item podle důsledku věty \emph{o uzávěru} platí $T \vdash A$ právě když je $T \cup \lbrace \neg A^{'} \rbrace$ sporná teorie ($A^{'}$ značí uzávěr $A$).
\item pokud je $T \cup \lbrace \neg A^{'} \rbrace$ sporná, tak není splněna v žádném modelu (dle části 2. věty o úplnosti), takže ve všech modelech 
$T$ musí být splněna $T \cup \lbrace A^{'} \rbrace$
\item jestli je uzávěr $A$ splněn ve všech modelech, tak to znamená, že $A$ je taky splněna ve všech modelech, tedy: $T \models A$ \qed
\end{itemize}
\end{remark}

\paragraph{Rozšiřování.} Jazyk $L'$ nazveme rozšířením jazyka $L$, pokud nový jazyk obsahuje všechno, co obsahoval ten původní (funkční symboly a predikáty).

\begin{definition}[Rozšíření teorie]
Mějme teorii $T$ v jazyce $L$. Teorii $T'$ v jazyce $L'$ nazveme rozšířením teorie $T$, pokud:
\begin{enumerate}
\item jazyk $L'$ je rozšířením $L$
\item pro každou formuli $A$ z jazyka $L$:
\[
(T \vdash A)\ \Rightarrow\ (T' \vdash A)
\]  
\end{enumerate}
Pokud platí také opačná implikace:
\[
(T \vdash A)\ \Leftarrow\ (T' \vdash A)
\]
hovoříme o \emph{konzervativním rozšíření}.
\end{definition}

\begin{remark}[Rozšíření o konstantní symbol]
je vždy konzervativní rozšíření. (plyne z věty o konstantách)
\end{remark}

\paragraph{Rozšíření teorie o definici funkcí a predikátů}
Je rozdíl mezi \emph{zavedením funkčního symbolu} a \emph{definicí funkčního symbolu} (příklad):
\begin{itemize}
\item zavedení: \emph{Nechť je $p$ prvočíslo větší než $x$.} (Takových prvočísel je mnoho - víme, že ke každému $x$ nějaké $p$ najdeme)
\item definice: \emph{Nechť je $p$ nejmenší z prvočísel, která jsou větší než $x$.} (Ke každému $x$ máme právě jedno $p$)
\end{itemize}
Oběma způsoby můžeme získat konzervativní rozšíření teorie o funkční symbol.

\begin{theorem}[Definice predikátového symbolu]
Máme teorii $T$ s jazykem $L$. Formule $D$ v jazyce $L$ obsahuje volné proměnné $x_1,\ldots,x_n$. Pokud $T^{'}$ vznikne z $T$
rozšířením jazyka $L$ o $n$-ární predikátový symbol $p$ a přidáním axiomu:
\[
p(x_1,\ldots,x_n) \leftrightarrow D
\]
potom je $T^{'}$ kozervativní rozšíření $T$. Navíc lze ke každé fli. $B^{'}$ sestrojit $B$ v jazyce $L$ tak, že: $T^{'} \vdash B^{'} \leftrightarrow B$
\end{theorem}

\begin{theorem}[Henkin]
Ke každé teorii $T$ lze sestrojit teorii $T_{H}$, která je konzervativním rozšířením $T$.
\end{theorem}

\begin{theorem}[Zavedení funkčního symbolu]
Máme teorii $T$ s jazykem $L$. Formule $A$ v jazyce $L$ obsahuje volné proměnné $x_1,\ldots,x_n,y$ a platí:
\[
T \vdash (\exists y)A
\]
Pokud $T^{'}$ vznikne z $T$ rozšířením jazyka $L$ o funkční symbol $f$ a přidáním axiomu:
\[
A_{y}[f(x_1,\ldots,x_n)]
\]
potom je $T^{'}$ konzervativní rozšíření $T$.
\end{theorem}

\begin{theorem}[Definice funkčního symbolu]
Máme teorii $T$ s jazykem $L$. Formule $D$ v jazyce $L$ obsahuje volné proměnné $x_1,\ldots,x_n,y$ a platí:
\[
\begin{array}{rcl}
T & \vdash & (\exists y)D\\
T & \vdash & D \rightarrow (D_{y}[t] \rightarrow y=t)
\end{array}
\]
Pokud $T^{'}$ vznikne z $T$ rozšířením jazyka $L$ o funkční symbol $f$ a přidáním (definujícího) axiomu:
\[
y=f(x_1,\ldots,x_n) \leftrightarrow D
\]
potom $T^{'}$ je konzervativní rozšíření $T$. Navíc pro každou fli. $B^{'}$ lze sestrojit $B$ v jazyce $L$ tak, že:
$T^{'} \vdash B^{'} \leftrightarrow B$
\end{theorem}

\section{Pátá část}

\paragraph{Míry výpočtové složitosti}
Zkoumáme prostorovou a časovou složitost algoritmů vzhledem k velikosti vstupních dat.
\begin{itemize}
\item časová složitost - počet elementárních kroků, které musí algoritmus vykonat během výpočtu
\item prostorová složitost - objem paměti, kterou algoritmus během výpočtu zabere
\end{itemize}

V teorii složitosti se využívá výpočetní model Turingova stroje - časová složitost je potom počet kroků TS a prostorová složitost odpovádá
počtu zabraných buněk na pásce.

\paragraph{Třídy složitosti}
Pokud přeformulujeme problémy jako rozhodovací problémy rozpoznávání jazyků pomocí TS, můžeme definovat množiny jazyků:
\begin{itemize}
\item $DTIME(f(n))$ - množina jazyků, které lze rozpoznat pomocí DTS v čase $O(f(n))$
\item $DSPACE(s(n))$ - množina jazyků, které lze rozpoznat pomocí DTS v prostoru $O(s(n))$ 
\end{itemize}
Podobně pro NTS zavádíme množiny $NTIME$ a $NSPACE$.

Složitostní třídy $P,NP,PSPACE,NPSPACE,LOGSPACE$ jsou potom definovány jako:
\begin{itemize}
\item $P = \bigcup_{c > 0} DTIME(n^{c})$
\item $NP = \bigcup_{c > 0} NTIME(n^{c})$
\item $PSPACE = \bigcup_{c > 0} DSPACE(n^{c})$
\item $NPSPACE = \bigcup_{c > 0} NSPACE(n^{c})$
\item $LOGSPACE = DSPACE(log(n))$
\end{itemize}

\paragraph{Vztahy mezi složitostními třídami.}  Platí:
\begin{itemize}
\item $P \subseteq NP \subseteq PSPACE = NPSPACE$
\item $LOGSPACE \subseteq PSPACE$
\end{itemize}
Přitom rovnost $PSPACE = NPSPACE$ plyne ze Savičovy věty.

\paragraph{NP-těžkost a NP-úplnost}

\paragraph{Složitost algoritmů v AI}

\paragraph{Prohledávání.} Základní stromové prohledávací algoritmy v AI:
\begin{itemize}
\item BFS - prohledávání do šířky
\item DFS - prohledávání do hloubky
\item Uniform Cost Search - prohledávání podle ceny cesty (BFS s prioritní frontou)
\item LDFS - omezené DFS (prohledávání omezené hloubkou)
\item IDFS - iterované prohlubování
\item Bi-directional search - prohledává se od startu i od cíle
\end{itemize}

\paragraph{Rezoluční odvozování}
Rezoluce patří mezi NP-úplné problémy - algoritmus v každém kroku nedeterministicky vybírá dvojici klauzulí pro rezoluci, dokud neodvodí prázdnou klauzuli (spor), nebo nenajde pevný bod (už nepřibývají žádné nové klauzule). 

\end{document}

