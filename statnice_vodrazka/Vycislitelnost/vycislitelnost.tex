\documentclass[a4paper]{article}      % Comments after  % are ignored
\usepackage{amsmath,amssymb,amsfonts} % Typical maths resource packages
\usepackage[utf8]{inputenc}
\usepackage{czech}
\usepackage{graphicx}

\newtheorem{theorem}{Věta}[section]
\newtheorem{lemma}[theorem]{Lemma}
\newtheorem{proposition}[theorem]{Předpoklad}
\newtheorem{corollary}[theorem]{Důsledek}

\newenvironment{proof}[1][Důkaz]{\begin{trivlist}
\item[\hskip \labelsep {\bfseries #1}]}{\end{trivlist}}
\newenvironment{definition}[1][Definice]{\begin{trivlist}
\item[\hskip \labelsep {\bfseries #1}]}{\end{trivlist}}
\newenvironment{example}[1][Příklad]{\begin{trivlist}
\item[\hskip \labelsep {\bfseries #1}]}{\end{trivlist}}
\newenvironment{remark}[1][Pozorování]{\begin{trivlist}
\item[\hskip \labelsep {\bfseries #1}]}{\end{trivlist}}

\newcommand{\qed}{$\blacksquare$}

\begin{document}


\section*{Důležité poznatky}

\begin{definition} \emph{Kleeneho predikát} je definován takto:
\[T_1(e,x,y)\]
kde $e$ je kód funkce jedné proměnné. Když budeme simulovat výpočet této funkce při vstupu $x$ na TS, zastaví se přesně po $y_0$ krocích s výsledkem $y_1$
(kde $<y_0,y_1> = y$ je zakódovaná dvojice čísel).\\
Predikát lze obecně definovat i pro funkce $n$ proměnných:
\[T_1(e,x_1,\ldots,x_n,y)\]     
\end{definition}

\begin{definition} \emph{Operátor minimalizace pro predikáty}\\
Mějme predikát $P(x_1,\ldots,x_n,y)$ s charakteristickou funkcí $c_P(x_1,\ldots,x_n,y)$.
Potom definujeme funkci:
\[\mu_y P(x_1,\ldots,x_n,y) =
\left\{
\begin{array}{l}
y\quad pokud\quad c_P(x_1,\ldots,x_n,y)\downarrow = 0\quad \wedge\\
\forall z < y: \quad c_P(x_1,\ldots,x_n,z)\downarrow \neq 0\\
\\
nedef.
\end{array}
\right.
\]
Jde vlastně o odvození nové funkce z charakteristické funkce daného predikátu pomocí operátoru
minimalizace.
\end{definition}

\begin{remark} \emph{Minimalizace Kleeneho predikátu}\\
\[\mu_y T_1(e,x,y) = y = <y_0,y_1>\]
V tomto výrazu odpovídá $y$ první nalezené dvojici $<$ počet kroků,výsledek$>$, ke které dospěje TS simulující výpočet funkce s číslem $e$ nad vstupem $x$.       
\end{remark}

\begin{definition} \emph{Funkce U a její ekvivalentní značení}\\
\[U(\mu_y T_1(e,x,y))\]
je \emph{univerzální funkce pro ČRF jedné proměnné}. Jako svůj výsledek vrací výstup
funkce s číslem $e$ pro vstup $x$. Spočte ho tak, že rozkóduje dvojici $y = <y_0,y_1>$
(výsledek $\mu_y T_1(e,x,y)$)
a vrátí $y_1$. Lze zapsat jako:
\begin{itemize}
\item $\Psi_1(e,x)$
\item $\varphi_e(x)$
\end{itemize}
Zobecněné verze pro $n$ proměnných:
\begin{itemize}
\item $U(\mu_y T_n(e,x_1,\ldots,x_n,y))$
\item $\Psi_n(e,x_1,\ldots,x_n)$
\item $\varphi_e(x_1,\ldots,x_n)$
\end{itemize}
\end{definition}

\begin{theorem} \emph{Kleene}
Existuje PRF U, a platí $\forall k \geq 1$:
\begin{itemize}
\item $\exists\quad PRF\quad s_k$ pro $k+1$ proměnných (fixace prvních $k$ proměnných)
\item $\exists\quad PRP\quad T_k$ pro $k+2$ proměnných (Kleeneho predikát)
\item $\exists\quad CRF\quad \Psi_k$ pro $k+1$ proměnných (univerzální funkce)             
\end{itemize}
takové, že:
\begin{enumerate}
\item $\Psi_k$ je univerzální ČRF pro třídu ČRF $k$ proměnných. Z odvození dané ČRF lze efektivně zjistit číslo $e$ a z $e$ lze získat nazpět odvození (Gödelovská enumerace).
\item $\Psi_k (e,x_1,\ldots,x_k) \simeq U(\mu_y(T_k(e,x_1,\ldots,x_k,y)))$
\item $s_k$ jsou prosté funkce rostoucí ve všech proměnných
\item $T_k(e,x_1,\ldots,x_k,y)\wedge T_k(e,x_1,\ldots,x_k,z) \Rightarrow (y=z)$
\item $\Psi_{m+n} (e,y_1,\ldots,y_m,x_1,\ldots,x_n) \simeq
\Psi_n (s_m(e,y_1,\ldots,y_m),x_1,\ldots,x_n)$ (s-m-n věta)   
\end{enumerate}
\end{theorem}

\begin{definition} \emph{Ackermannova funkce}
\begin{eqnarray*}
A(0,x) & = &
\left\{
\begin{array}{ll}
1 & (x = 0)\\
2 & (x = 1)\\
x+2 & (x > 1)
\end{array}
\right.\\
A(y+1,0) & = & 1\\
A(y+1,x+1) & = & A(y,A(y+1,x))
\end{eqnarray*}
\end{definition}

\section{První část}

\subsection{Algoritmicky vyčíslitelné funkce}
Abychom mohli zkoumat co všechno lze spočítat, je třeba nejdřív zvolit výpočetní model.
Existuje více výpočetních modelů, o kterých lze ukázat, že jsou stejně silné. Podle Church-Turingovy teze následující modely nějakým způsobem formalizují pojem \emph{algoritmu}:
\begin{itemize}
	\item Turingův stroj
	\item Částečně rekurzivní funkce
	\item (Lambda kalkulus)
	\item (Postovy produkční systémy)
\end{itemize}

\paragraph{Church-Turingova teze} říká, že ke každému algoritmu existuje ekvivalentní turingův stroj.

Použijeme-li formalizaci využívající turingův stroj, můžem říct, že funkce $f: N \rightarrow N$ je \emph{algoritmicky vyčíslitelná}, pokud existuje turingův stroj, který ji vyčísluje.  

Formalizace pomocí ČRF se zavádí pomocí primitivních funkcí (nulová, následník, vydělení $i$-té složky) a operátorů (substituce, primitivní rekurze, minimalizace). V rámci této formalizace se zavádí třídy PRF, ORF a ČRF (viz. níže). Třída ČRF je nejobecnější z nich a lze ukázat, že je ekvivalentní s třídou všech \emph{algoritmicky vyčíslitelných funkcí}.

Neformálně lze říct že:
\begin{itemize}
\item PRF jsou ekvivalentní programům, které nepoužívají \verb+while+ cyklus.
\item ORF jsou ekvivalentní programům, které pro každý vstup skončí svůj výpočet.
\item ČRF jsou ekvivalentní programům, které používají i \verb+while+ cyklus a může se tedy stát, že nikdy neskončí svůj výpočet.
\end{itemize}

\subsection{Ekvivalence definic pro algoritmicky vyčíslitelné funkce}
Využíváme faktu, že všechny TS lze efektivně očíslovat pomocí \emph{G\"{o}delovy enumerace}. Abychom to mohli udělat, je třeba nejdřív
zakódovat TS do binárního řetězce. Binární řetězce, které nekódují žádný TS bereme jako stroj který nic neudělá a hned skončí.

\paragraph{Ekvivalence ČRF a TS}
\begin{itemize}
	\item $\Rightarrow$: sestavíme TS pro primitivní funkce a ukážeme, že podle odvozovacích pravidel vytvoříme z více malých TS zase TS.
	Existuje-li odvození pro ČRF $f$, umíme podle něj sestavit TS.
	\item $\Leftarrow$:
		\begin{itemize}
			\item Pokud je funkce turingovsky vyčíslitelná, existuje TS $M_{e}$, který ji vyčísluje.
			\item Výpočet $M_{e}$ je posloupnost konfigurací. Každou z nich umíme zakódovat do binárního řetězce.
			\item Přechod mezi konfiguracemi představuje lokální změnu, která jde vyjádřit pomocí PRF.
			Tím se v PRF zachytí rozhodovací strom výpočtu $M_{e}$. 
			\item Použijeme primitivní rekurzi k sestavení ORF $Comp(s,k)$ která vydá kód stavu stroje,
			který začne ve stavu $s$ a skončí po $k$ krocích. 
			\item Pro dané slovo $x$ a stroj $M_{e}$ umíme pomocí operátoru minimalizace zjistit po kolika krocích se
			$M_{e}$ dostane do nějaké koncové konfigurace.
			\item Dohromady potom můžeme sestavit ČRF, která vyčísluje totéž co stroj $M_{e}$. Uděláme to tak, že z výsledku
			$Comp(s_{x},k_{f})$, kde $s_{x}$ je stav stroje na počátku výpočtu nad slovem $x$ a
			$k_{f}$ je počet kroků vedoucích do koncového stavu získaný operátorem minimalizace pro výpočet nad slovem $x$,
			vytáhneme koncové slovo z poslední konfigurace.  
		\end{itemize}
\end{itemize}

\section{Druhá část}

\subsection{Primitivní funkce}
\begin{enumerate}
\item Nulová funkce:\\
$\forall x \in \mathbf{N}: \quad \mathbf{o}(x) = 0$
\item Funkce následníka:\\
$\forall x \in \mathbf{N}: \quad \mathbf{s}(x) = x + 1$
\item Vybrání $j$-tého prvku:\\
$n,j \in \mathbf{N}, 1 \leq j \leq n, x_{1},\ldots,x_{n}\in \mathbf{N}: \mathbf{I}_{n}^{j}(x_{1},\ldots,x_{n}) = x_{j}$
\end{enumerate}

\subsection{Operátory}
\begin{enumerate}
\item Substituce:
\[m,n \in \mathbf{N}, f:\mathbf{N}^{m}\rightarrow \mathbf{N}, g_{i}: \mathbf{N}^{n} \rightarrow \mathbf{N}, i = 1,\ldots,m:\]
\[\mathbf{S}_{n}^{m}(f,g_{1},\ldots,g_{m}) = f(g_{1}(\vec{x}),\ldots,g_{m}(\vec{x}))\]
Jde o analogii složení velkého programu z podprogramů.
\item Primitivní rekurze:
\[\forall n \in \mathbf{N}, f: \mathbf{N}^{n-1} \rightarrow \mathbf{N}, g: \mathbf{N}^{n+1}\rightarrow \mathbf{N}, h:\mathbf{N}^{n}\rightarrow \mathbf{N}, \vec{x} \in \mathbf{N}^{n-1}\]
\[\mathbf{R}_{n}(f,g) = h\]
funkce $h$ je přitom definována rekurzivně:\\ 
\[h(0,\vec{x}) = f(\vec{x})\]
\[h(i+1,\vec{x}) = g(i,h(i,\vec{x}),\vec{x})\]
Jde o analogii \emph{for} cyklu. Kdybychom $for$ cyklus definovali rekurzí, tak by funkce $f$
sloužila jako ukončovací podmínka rekurze a $g$ by se prováděla při každém novém rekurzivním volání.
\item Minimalizace:
\[\forall n \in \mathbf{N}, f:\mathbf{N}^{n+1} \rightarrow \mathbf{N}, h:\mathbf{N}^{n} \rightarrow \mathbf{N}\]
\[\mathbf{M}_{n}(f) = h\]
kde pro funkci $h$ platí:\\
\[h(x_{1},\ldots,x_{n})\downarrow = z, z \in \mathbf{N}\]
\[\Leftrightarrow\]
\[f(x_{1},\ldots,x_{n},z)\downarrow = 0 \wedge \forall j < z : f(x_{1},\ldots,x_{n},j)\downarrow \neq 0\]
Jde o analogii \emph{while} cyklu. Kód v \verb#C++# pro funkci $f$ se dvěma proměnnými by mohl vypadat takto:
\begin{verbatim}
int f(x1,y);

int h(x1)
{
   int z = 0;

   while(f(x1,z) != 0)
        ++z;
	
   return z;
}
\end{verbatim}
\end{enumerate}

\begin{definition} \emph{PRF,ORF,ČRF}
\begin{itemize}
\item \emph{Částečně rekurzivní funkce} jsou odvozeny ze základních funkcí pomocí operátorů.
\item \emph{Obecně rekurzivní funkce} jsou ty ČRF, které jsou všude definované.
\item \emph{Primitivně rekurzivní funkce} jsou takové ORF, při jejichž odvození nebylo použito odvozovací pravidlo minimalizace.
\end{itemize}
\end{definition}

\begin{theorem}[Hierarchie ČRF] pro třídy ČRF, ORF a PRF platí:
\[
PRF \subsetneq ORF \subsetneq CRF
\]
\end{theorem}

\begin{proof}
První ostrá inkluze $PRF \subsetneq ORF$:\\
Ukážeme, že Ackermanovu funkci nelze odvodit bez operátoru minimalizace.
V důkazu se zavádí hloubka funkcí podle počtu vnořených \verb+for+-cyklů a ukáže se,
že žádný konečný počet vnoření pro ackermanovu funkci nestačí.
Druhá ostrá inkluze $ORF \subsetneq CRF$:\\
Ukážeme existenci funkce, která je ČRF ale není ORF. Provedeme minimalizaci funkce $g(x,y)=y+1$, která \emph{nikdy nevrací 0}:
\[
M(g(x,y)) = h(x): h(x)\downarrow = z \Leftrightarrow g(x,z)\downarrow = 0 \wedge \forall y \leq z g(x,y)\downarrow \neq 0
\]
Hledaná funkce (ČRF co není ORF) je potom $h$.
\end{proof}

\section{Třetí část}

\begin{definition}[Rekurzivní predikát]
je každý predikát $P(x)$, pro který existuje nějaká \textbf{ORF} $c_{P}$ (tzv. charakteristická funkce), že platí:
\[
c_{P}(x)= \left\{
\begin{array}{l}
1 \Leftrightarrow P(x) = true\\
0 \Leftrightarrow P(x) = false
\end{array}
\right.
\]
\end{definition}

\begin{definition}[Rekurzivně spočetný predikát]
je každý predikát $P(x)$, pro který existuje nějaká \textbf{ČRF} $c_{P}$ (tzv. charakteristická funkce), že platí:
\[
c_{P}(x)\downarrow = \left\{
\begin{array}{l}
1 \Leftrightarrow P(x) = true\\
0 \Leftrightarrow P(x) = false
\end{array}
\right.
\]
Pokud $f(x)\uparrow$, tak nevíme o pravdivosti predikátu nic.
\end{definition}

\begin{definition}[Rekurzivní množina]
je každá množina $M$, ke které existuje \emph{rekurzivní predikát} $P_{M}(x)$ definovaný následovně:
\[
\forall x: x\in M \Leftrightarrow P_{M}(x)
\]
Každá rekurzivní množina je také definičním oborem nějaké ORF.
\end{definition}

\begin{definition}[Rekurzivně spočetná množina]
je každá množina $M$, ke které existuje \emph{rekurzivně spočetný predikát} $P_{M}(x)$ splňující:
\[
\forall x: x\in M \Leftrightarrow P_{M}(x)
\]
\end{definition}

\begin{definition}[Graf funkce.] Pro funkci $f$ definujeme její graf jako množinu:
\[
\lbrace <x,y>\ |\ f(x)\downarrow = y \rbrace
\]
Kde $<x,y>$ je zakódovaná dvojice čísel $x$ a $y$.
\end{definition}

\begin{theorem}[O množinových operacích a RSM.]
Konjunkce a disjunkce zachovávají rekurzivní spočetnost. Mějme RSM $A$ a $B$. Potom platí:
\begin{itemize}
\item $A \cup B$ je \emph{rekurzivně spočetná množina}
\item $A \cap B$ je \emph{rekurzivně spočetná množina}
\end{itemize}
\end{theorem}

\paragraph{Ekvivalentní definice} pro rekurzivně spočetnou množinu $M$:\\
$M$ je rekurzivně spočetná právě když:
\begin{itemize}
\item $M$ je definičním oborem nějaké ČRF
\item existuje primitivně rekurzivní predikát $P(x,y)$, pro který platí:
\[
M = \lbrace x\ |\ (\exists y)P(x,y)\rbrace
\]
\item $M$ je grafem nějaké ČRF
\end{itemize}

\paragraph{Očíslování rekurzivně spočetných množin.} Jestliže $e$ je G\"{o}delovo číslo nějakého TS (nějaké ČRF), potom
\[
W_{e} = \lbrace x | \varphi_{e}(x)\downarrow \rbrace
\] 
je $e$-tá rekurzivně spočetná množina.

\begin{theorem}[Post]
Množina $M$ je \emph{rekurzivní} právě když $M$ i její doplněk $\bar{M}$ jsou \emph{rekurzivně spočetné}.
\end{theorem}

\begin{theorem}[Věta o selektoru]
Pro každý rekurzivně spočetný predikát $R(x,y)$ existuje taková ČRF $f$, že platí:
\[
\exists y:\ R(x,y) \Leftrightarrow f(x)\downarrow = y
\]
\end{theorem}

\begin{definition}[Převoditelnost.]
\begin{itemize}
\item Množina $A$ je \emph{1-převoditelná} na $B$ právě když existuje \emph{prostá} ORF $f$ tak že:
\[
\forall x \in A \Leftrightarrow f(x) \in B
\]
Značení: $A \leq_{1} B$ 
\item Množina $A$ je \emph{m-převoditelná} na $B$ právě když existuje nějaká ORF $f$ tak, že platí výše uvedená podmínka.
Značení: $A \leq_{m} B$
\end{itemize}
\end{definition}

\begin{definition}[Úseková funkce]
je taková funkce, jejímž definičním oborem je počáteční úsek (nebo celé) $\mathcal{N}$.
\end{definition}

\begin{theorem}[Rekurzivní množiny]
jsou právě obory hodnot rostoucích úsekových ČRF.
\end{theorem}

\begin{theorem}[Rekurzivně spočetné množiny]
jsou právě obory hodnot prostých úsekových ČRF.
\end{theorem}

\section{Čtvrtá část}

\subsection{Algoritmicky nerozhodnutelné problémy}

\begin{theorem} \emph{Halting problem}\\
Neexistuje algoritmus, který by pro daný turingův stroj $T$ a data $S$ rozhodnul, jestli $T(S)$ konverguje (skončí výpočet), nebo diverguje (neskončí výpočet).
\end{theorem}

\begin{proof}
Pro spor předpokládejme že takový algoritmus existuje. Označme ho $H$. Platí, že $\forall (T,S)$, kde $T$ je kód turingova stroje a $S$ jsou vstupní data, algoritmus $H$ skončí a dá odpověď:
\begin{equation}
H(T,S) =
\left\{
	\begin{array}{ll}
	1 & pokud\qquad T(S)\downarrow\\
	0 & pokud\qquad T(S)\uparrow\\
	\end{array}
\right.
\end{equation}
Pomocí algoritmu $H$ sestrojíme turingův stroj $A$ (který vždy zastaví, protože je odvozen od $H$).
Pokud $K$ je kód nějakého TS, chování stroje $A$ je definováno následovně:
\begin{eqnarray}
A(K) = 1\quad \Leftrightarrow\quad H(K,K) = 0\\
A(K) = 0\quad \Leftrightarrow\quad H(K,K) = 1 
\end{eqnarray}
Je-li $Q$ kód stroje $A$, snadno dojdeme ke sporu:
\[
A(Q) = 0 \Leftrightarrow H(Q,Q) = 1 \Leftrightarrow A(Q) = 1
\] 
\end{proof}

\paragraph{Halting problém jako množina:}
\[
K_0 = \lbrace <x,y>\ |\ \varphi_{x}(y)\downarrow \rbrace
\]
Jde o množinu dvojic $<x,y>$, takovou, že stroj s kódem $x$ se zastaví na datech $y$. Tato množina je rekurzivně spočetná, ale není rekurzivní.
Diagonálu problému zastavení lze zapsat jako:
\[
K = \lbrace x\ |\ \varphi_{x}(x)\downarrow \rbrace 
\]

Důkaz nerozhodnutelnosti (tedy částečné rekurzivity) $K_0$ lze provést převodem na $K$ o které předtím dokážeme, že je částečně rekurzivní.

\begin{theorem} \emph{Cantorova diagonální metoda}\\
Mějme posloupnost množin $\lbrace A_n \rbrace_n$.
Potom pro množinu $B = \lbrace x_n \| x_n \notin A\rbrace$ platí: $\forall n:\quad B\neq A_n$.
\end{theorem}

\section{Pátá část}

\subsection{Věty o rekurzi}
\paragraph{Co říká věta o rekurzi?} funkce $f$ ve větě o rekurzi dělá to, že vezme číslo programu a přepíše ho na jiné číslo tak,
že program s tím novým číslem počítá totéž. No a to je přeci super ne? Teď víme, že lze "portovat" programy mezi různými jazyky a že
může existovat víc implementací, které dělají totéž.

\begin{theorem}[Věta o rekurzi]
Pro libovolnou ČRF $f$ existuje číslo $a$ tak, že pro každý vstup $x$ platí:
\[
\varphi_{f(a)}(x) \simeq \varphi_{a}(x)
\]
\end{theorem}

\begin{proof}
Sestavíme TS s číslem $e$, který poté, co dostane na vstupu $z$ a $x$ pracuje následovně:
\begin{enumerate}
\item spočítá $s(z,z)$ (funkce $s$ je z Kleeneho věty)
\item spustí $f$ na to  co vzešlo z $s(z,z)$ (tedy $f(s(z,z))$) a čeká...
\item pokud se dočká, tak zavolá univerzální stroj s číslem $f(s(z,z))$ na vstup $x$
\end{enumerate}
Platí:
\[
\varphi_{f(s(z,z))}(x) \simeq \psi_{2}(e,z,x) \simeq^{s-m-n} \psi_{1}(s(e,z),x) \simeq \varphi_{s(e,z)}(x)
\]
No a teď stačí vytáhnout trumfy: $a=s(e,e)$ a $e=z$. Takže věta platí.
\end{proof}

\begin{remark}
Program s číslem $a$ počítá déle:
\begin{enumerate}
\item vezme $e$, a spočítá $s(e,e)$ (tedy svoje vlastní číslo $a$)
\item zavolá $f(a)$ aby věděl jaký program má zavolat pro výpočet nad vstupem $x$
\item zavolá $\varphi_{f(a)}(x)$
\end{enumerate}
Program s číslem $f(a)$ už rovnou volá UTS na vstup $x$.
\end{remark}

\begin{definition}[Produktivní množina]
\end{definition}

\subsection{Aplikace vět o rekurzi}

\begin{theorem}[Rice]
Buď $\mathcal{A}$ \emph{netriviální}(není prázdná, ani neobsahuje všecny ČRF) třída částečně rekurzivních funkcí. Dále buď
\[
A_{\mathcal{A}}= \lbrace x\ |\ \varphi_{x} \in \mathcal{A}\rbrace
\]
množina jejich čísel. Platí tvrzení: $A_{\mathcal{A}}$ není rekurzivní množina. 
\end{theorem}

\begin{proof}
Sporem: předpokládejme, že je $A_{\mathcal{A}}$ rekurzivní.
Potom existuje ORF $f$, která je charakteristickou funkcí množiny $A_{\mathcal{A}}$.
Dále mějme čísla $a \in A_{\mathcal{A}}$ a $b \notin A_{\mathcal{A}}$ (nějaké takové existují, protože je $\mathcal{A}$ netriviální).

Nyní zkonstruujeme ORF $h$:
\[
h(x) = \left\{
\begin{array}{lcl}
a & \Leftrightarrow & f(x) = 0\\
b & \Leftrightarrow & f(x) = 1\\
\end{array}
\right.
\]

Podle \textbf{věty o rekurzi} má ale funkce $h$ pevný bod $p$ takový, že:
\[
\varphi_{p}(x) \simeq \varphi_{h(p)}(x)
\]
takže funkce s čísly $p$ a $h(p)$ vyčíslují totéž a měly by obě být uvnitř $\mathcal{A}$ nebo mimo. Což je ve sporu s tím, že
pokud je $p$ uvnitř $A_{\mathcal{A}}$ tak $h(p)$ je mimo a naopak.
\end{proof}

\paragraph{Co říká Riceova věta?} O tom, jestli \emph{dva programy počítají totéž} nelze rozhodnout algoritmicky.
Třída $\mathcal{A}$ je zde třída programů které počítají totéž a množina $A_{\mathcal{A}}$ je množinou jejich kódů.
Pokud by byla tato množina rekurzivní, dokázali bychom efektivně rozhodnout pro dané dva programy, jestli počítají totéž.
Pan Rice říká, že to nejde.  

\section{Šestá část}

\subsection{Gödelovy věty}

\begin{definition}[Rekurzivní neoddělitelnost]
dvojice množin $A,B: A \cap B = \emptyset$ je \emph{rekurzivně neoddělitelná}, pokud neexistuje žádná rekurzivní množina $M: A\subseteq M$ pro kterou: $M \cap B = \emptyset$ 
\end{definition}

\begin{definition}[Efektivní neoddělitelnost]
dvojice množin $A,B: A \cap B = \emptyset$ je \emph{efektivně neoddělitelná}, pokud exisuje ČRF $\varphi$ taková, že:
\[
\left.
\begin{array}{r}
\exists x: A\subseteq W_{x}\\
\exists y: B\subseteq W_{y}\\
W_{x} \cap W_{y} = \emptyset
\end{array}
\right\} \Rightarrow \varphi(x,y)\downarrow\ \wedge\ \varphi(x,y) \notin W_{x} \cup W_{y}
\] 
\end{definition}

\begin{definition}[Základní aritmetická síla]
Jazyk prvního řádu má \emph{základní aritmetickou sílu} pokud obsahuje:
\begin{itemize}
\item $\bar{0}$ - literál pro 0
\item $\bar{1}$ - literál pro 1
\item $+,\times$ - funkční symboly
\item konečně mnoho axiomů
\end{itemize}
\end{definition}

\begin{definition}[Axiomatizovatelnost]
Teorie $T$ je axiomatizovatelná, pokud množina všech dokazatelných formulí je rekurzivně spočetná.
\end{definition}

\begin{theorem}[Gödelovy věty o neúplnosti]
Jestliže má $T$ základní aritmetickou sílu a je bezesporná, potom:
\begin{enumerate}
\item množina formulí dokazatelných v $T$ není rekurzivní
\item Pokud je $T$ axiomatizovatelná, potom navíc:
	\begin{enumerate}
		\item existuje formule $F$, kterou v $T$ nelze rozhodnout
		\item v $T$ nelze dokázat vlastní bezespornost 
	\end{enumerate}
\end{enumerate}
\end{theorem}

\end{document}

