\subsection{Zabezpečení, systém práv, správa uživatelských účtů}

TODO: tohle je jenom copy \& paste z Wiki

\subsubsection*{Zabezpečení}

\begin{pitemize}
  \item procesy běží v uživatelském režimu s omezenými možnostmi, při systémových voláních se proces přepne do režimu jádra
  \item preemptivní plánování, priority
  \item firewall - iptables (Linux), pf (OpenBSD), ipfw (FreeBSD), FW-1 (Solaris), ipfilter aj.
  \item vypnutí nepotřebných služeb (daemonů)
  \item zálohování
  \item sledování logů
  \item nmap
  \item Kerberos
  \item chroot
  \item sifrovani disku
  \item sifrovani komunikace
  \item pouceni uzivatelu
  \item silna hesla 
\end{pitemize}

\subsubsection*{Systém práv}

\begin{pitemize}
  \item každý soubor a proces mají vlastníka a skupinu
  \item práva pro vlastníka, skupinu a ostatní - čtení, zápis, spouštění
  \item setuid, setgid - propůjčení práv vlastníka/skupiny při spuštění programu
  \item setgid pro adresáře - nové soubory budou mít stejnou skupinu jako adresář
  \item sticky bit pro adresáře - práva k souborům mají jen vlastníci souborů a nikoliv vlastníci adresáře
  \item uživatel root
  \item reálné a efektivní UID/GID u běžících procesů
  \item chmod, chown, chgrp, umask 
\end{pitemize}

\subsubsection*{Správa uživatelských účtů}

\begin{pitemize}
  \item /etc/passwd - seznam uživatelů - login, UID, GID, plné jméno, domovský adresář, shell
  \item /etc/group - skupiny - název, GID, seznam členů
  \item /etc/shadow - zašifrovaná hesla (hash) - může číst pouze root
  \item useradd, userdel, usermod, groupadd, groupdel, groupmod, passwd
\end{pitemize}
