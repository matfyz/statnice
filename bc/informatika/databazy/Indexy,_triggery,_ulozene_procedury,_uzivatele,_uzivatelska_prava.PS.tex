\subsection{Indexy, triggery, uložené procedury, uživatelé}

TODO: pořádná definice indexu

\begin{obecne}{Index}
Index je obvykle definován výběrem tabulky a jejího konkrétního sloupce (nebo sloupců), nad kterými si designér databáze přeje dotazování urychlit; dále pak technickým určením typu. Chování a způsoby uložení indexů se mohou významně lišit podle použité databázové technologie.
Výjimku mohou tvořit například full-textové indexy, které jsou v některých případech (nerelační databáze typu Lotus Notes) definovány nad celou databází, nikoliv nad konkrétní tabulkou. 
\end{obecne}

\begin{obecne}{Použití indexu}
Na první pohled by se mohlo zdát, že čím víc indexů, tím lepší chování databáze a že po vytvoření indexů pro všechny sloupce všech tabulkách dosáhneme maximálního zrychlení. Tento přístup naráží bohužel na dva zásadní problémy: 
\begin{penumerate}
    \item Každý index zabírá v paměti vyhrazené pro databázi nezanedbatelné množství místa (vzhledem k paměti vyhrazené pro tabulku). Při existenci mnoha indexů se může stát, že paměť zabraná pro jejich chod je skoro stejně velká, jako paměť zabraná jejími daty - zvláště u rozsáhlých tabulek (typu faktových tabulek v datovém skladu) může něco takového být nepřijatelné. 
    \item Každý index zpomaluje operace, které mění obsah indexovaných sloupců (například SQL příkazy UPDATE, INSERT). To je dáno tím, že databáze se v případě takové operace nad indexovaným sloupcem musí postarat nejen o změny v datech tabulky, ale i o změny v datech indexu. 
\end{penumerate}
\end{obecne}

\begin{obecne}{Typy indexů}
Indexy mohou mít svůj typ, který blíže určuje, jakým způsobem má být přistupování k datům tabulky optimalizováno. Označení se různí, ale nejčastěji je to:
\begin{pitemize}
    \item PRIMARY - Tento typ se v každé tabulce může vyskytovat nejvýše jednou. Definuje sloupec tabulky, který svou hodnotou jednoznačně identifikuje záznam. Ve většině případů se dodržuje konvence takový sloupec nazvat ID a jeho datový typ stanovit jako celé číslo (není-li potřeba jinak). Databázový server by měl být schopen nedopustit, aby byla do sloupce, k němuž se tento typ indexu vztahuje, byla vložena hodnota, která již v tabulce existuje (většinou takový pokus končí chybovou hláškou). 
    \item UNIQUE - Tento typ je podobný PRIMARY co do jednoznačnosti záznamu v tabulce (jak naznačuje i jeho název) a dopadu, který to na práci s databází má; ale může se vyskytovat u více sloupců tabulky. Podle účelu, ke kterému má tabulka sloužit, se občas definují indexy složené z více sloupců - potom opět nelze vložit záznam, který by již v této kombinaci někde v tabulce existoval. 
    \item INDEX - Definicí indexu tohoto typu je v tabulce zajištěna optimalizace vyhledávání podle sloupce, ke kterému se daný index váže. Většinou si databázový server vytvoří a nadále udržuje vnitřní seznam odkazů na řádky tabulky, seřazený podle hodnot sloupce, k němuž se váže. Udržování takto seřazené posloupnosti urychluje vyhledávání (je možno použít některé interpolační numerické metody), řazení i jiné zásahy do tabulky, které jsou omezeny podmínkou na dotyčné záznamy. 
    \item FULL-TEXT - Vytvořením tohoto indexu se databázový server bude snažit optimalizovat full-textové vyhledávání v daném sloupci u dané tabulky.
\end{pitemize}
\end{obecne}

\begin{obecne}{Triggery}
  Databázový trigger je programový kód, který je automaticky vykonán jako
  reakce na nějakou událost v určité databázové tabulce. Triggery mohou omezit
  přístup k určitým datům, provádět logování, nebo kontrolovat změny dat.

  Rozlišujeme dvě hlavní třídy triggerů a to \emph{řádkový trigger} a
  \emph{dotazový (statement) trigger}. Řádkový trigger můžeme definovat pro
  každý řádek tabulky, zatímco dotazový trigger se vykoná pouze jednou pro
  konkrétní databázový dotaz. Každá třída triggerů může být několika typů. Jsou
  \emph{before triggers} a \emph{after triggers}, což značí kdy má být trigger
  vykonán. Také se můžeme setkat s \emph{instead of triggers}, který je potom
  vykonán \emph{místo} dotazu kterým byl spuštěn. 

  Jaké události mohou trigger spustit se pochopitelně liší databázový systém od
  systému, ale existují tři typické události, které to mohou být:
  \begin{penumerate}
    \item INSERT~-- nový záznam je vložen do databáze,
    \item UPDATE~-- záznam je měněn,
    \item DELETE~-- záznam je mazán.
  \end{penumerate}
  Kromě těchto typických událostí může databázový systém umožňovat nastavovat
  triggery také na mazání, či vytváření celých tabulek, či dokonce přihlášení
  nebo odhlášení uživatele.

  Hlavní vlastnosti a efekty databázových triggerů jsou:
  \begin{pitemize}
    \item nepřijímají žádné parametry nebo argumenty,
    \item nemohou volat operace pro řízení transakcí COMMIT a ROLLBACK,
    \item mají přístup k datům, které budou měněny, je tedy možné vykonávat akce
    na základě nich,
    \item nemohou vracet záznamy,
    \item obtížně se ladí,
    \item mnoho triggerů nebo složité triggery mohou práci s databází velice
    zpomalit a navíc znepřehlednit.
  \end{pitemize}
\end{obecne}

\begin{obecne}{K čemu triggery používat?}
  Triggery se v databázích používají z několika důvodů, které mohou souviset s
  konzistencí dat, jejich údržbou, nebo mohou být způsob, kterým databáze
  komunikuje s okolím. Podívejme se na některá typická schémata:
  \begin{pitemize}
    \item \textbf{Konzistence dat}~-- Trigger může provést výpočet a na základě
    toho povolit nebo nepovolit změnu dat v databázi. Například trigger může zakázat
    smazání zákazníka z databáze v případě, kdy má u nás nějaký dluh a podobně.  
    \item \textbf{Logování}~-- Trigger může evidovat kdo, kdy a jak měnil data.
    Lze tak dohledat pracovníka, který zadal špatné údaje nebo zjistit, v kolik
    hodin došlo k včerejší uzávěrce.
    \item \textbf{Verzování dat}~-- Díky triggerům lze snadno naprogramovat
    aplikaci tak, aby jedna tabulka udržovala historii změn tabulky jiné. To lze
    s úspěchem použít třeba jako bezpečnostní mechanismus. 
    \item \textbf{Zasílání zpráv}~-- Trigger může spustit nějaký externí
    program nebo proces. Například může trigger autorovi poslat e-mail,
    pokud byl k jeho článku přidán příspěvek.
  \end{pitemize}
\end{obecne}

\begin{obecne}{Uložené procedury}
  Uložená procedura (anglicky stored procedure) je databázový objekt, který
  neobsahuje data, ale část programu, který se nad daty v databázi má vykonávat.

  Uložená procedura je především procedura. Jedná se o část programu, který je
  (nebo by aspoň měl být) jasně funkčně oddělený od svého okolí, má interface
  (seznam parametrů) pro komunikaci s jinými moduly programu. Může mít vlastní
  lokální proměnné neviditelné pro ostatní části programu. 

  Uložená procedura je uložená (rozuměj: uložená v databázi). To znamená, že se k
  ní lze chovat stejně jako ke každému jinému objektu databáze (indexu, pohledu,
  triggeru apod.). Lze jí založit, upravovat a smazat pomocí příkazů dotazovacího
  jazyka databáze (v případě relační databáze obvykle pomocí příkazů DDL SQL). 

  Pro psaní uložených procedur je obvykle používán specifický jazyk konkrétní
  databáze, který je rozšířením jejího dotazovacího jazyka (hezkým příkladem je
  databáze Oracle s procedurálním jazykem PL/SQL, který je rozšířením klasického
  dotazovacího jazyka SQL).
\end{obecne}

\begin{obecne}{Proč ukládat procedury?}
  \begin{pitemize}
      \item \textbf{Jednotné rozhraní}~-- Použití uložených procedur vychází z
      faktu, že většina operací nad daty v databázi probíhá stejně bez ohledu na
      to, kdo operaci provádí. Příklad: Pokud je třeba uložit do tabulky
      zákazníků nového zákazníka, tak se to z pohledu databáze děje stejně pro
      zákazníka internetového obchodu, pro zákazníka, kterého zadává pracovnice
      telefonického centra přes formulář programu napsaného například v C++ ,
      nebo pro zákazníky, kteří jsou vkládáni automaticky na základě textového
      reportu, který přijde každý den z \uv{kamenných} prodejních míst a je
      zpracováván pomocí programu napsaného v PowerBuilderu. Je tedy celkem
      dobrý důvod, aby existovala uložená procedura \uv{Zapiš nového zákazníka},
      kterou by mohly volat všechny tři výše uvedené aplikace - alternativou bez
      uložené procedury by bylo, že bych podobnou proceduru musel napsat ve
      třech verzích - jednou v C++, jednou v Power Builderu a jednou v rámci
      programu pro internetový obchod (třeba ASP nebo PHP). 
      \item \textbf{Skrytí datových operací}~-- Druhou výhodou použití uložených
      procedur je, že se nemusím (v programu na \uv{klientské} straně) zabývat
      tím, jak jsou data uložena v konkrétních tabulkách. V našem případě je mi
      jedno, jak si databáze uvnitř pamatuje zákazníky - prostě zadám jako
      parametr procedury jméno, příjmení, číslo kreditky a co si zákazník koupil
      - a databáze (resp. její uložená procedura) si to nějak přebere. Uložené
      procedury se v případě databázových aplikací staly základním kamenem pro
      realizaci architektury klient/server, kdy je na jedné straně (klientská
      část) realizována v běžném procedurálním programovacím jazyku komunikace s
      uživatelem (formuláře nebo třeba webové stránky) a na druhé straně
      (serverová část) je pomocí uložených procedur realizována správa dat v
      relační databázi. Obě části (klientská a serverová) mezi sebou komunikují
      přes co nejjednodušší rozhraní - voláním uložených procedur. 
  \end{pitemize}
\end{obecne}

TODO: uživatelé
