\subsection{Synchronizační primitiva, vzájemné vyloučení}

\subsubsection*{Pojmy}

\emph{Race conditions}: výsledek operace závisí na plánování

\emph{Vzájemné vyloučení (mutual exclusion)}: kritickou operaci provádí nejvýše jeden proces. Podmínky vzájemného vyloučení:
\begin{penumerate}
	\item Žádné dva procesy nemohou být najednou ve stejné kritické sekci
	\item Nemohou být učiněny žádné předpoklady o rychlosti nebo počtu CPU
	\item Žádný proces mimo kritickou sekci nesmí blokovat jiný proces
	\item Žádný proces nesmí čekat nekonečně dlouho v kritické sekci
\end{penumerate}

\emph{Kritická sekce}: část programu, kde se provádí kritická operace

Metody dosáhnutí vzájemného vyloučení: aktivní čekání (busy waiting) a pasivní čekání/blokování.

\subsubsection*{Aktivní čekání}
\emph{Vlastnosti}: spotřebovává čas procesoru, vhodnější pro předpokládané krátké doby čekání, nespotřebovává prostředky OS, rychlejší.

Je možné použít např. \emph{zakázání přerušení} (vhodné pro jádro OS). Používání \emph{zámků} nefunguje:
\begin{verbatim}
int lock;
void proc(void) {
  for (;;) {
    nekritická_sekce();
    while (lock != 0);
    lock = 1;
    kritická_sekce();
    lock = 0;
  }
}
\end{verbatim}
ale \emph{důsledné střídání} ano (to ale porušuje podmínku 3.)

\begin{verbatim}
int turn = 0;

void p1(void)                void p2(void)
{                            {
  for (;;) {                   for (;;) {
    while (turn != 0);           while (turn != 1);
    kritická_sekce();            kritická_sekce();
    turn = 1;                    turn = 0;
    nekritická_sekce();          nekritická_sekce();
  }                            }
}                            }
\end{verbatim}

\emph{Petersonovo řešení}:
\begin{verbatim}
#define N 2                   /* počet procesů */
 
int turn;
int interested[N];            /* kdo má zájem */
 
void enter_region(int proc) { /* proc: kdo vstupuje */
int other;
 
  other = 1-proc;             /* číslo opačného procesu */
  interested[proc] = TRUE;    /* mám zájem o vstup */
  turn = proc;                /* nastav příznak */
  while (turn == proc && interested[other] == TRUE);
}
 
void leave_region(int proc) { /* proc: kdo vystupuje */
  interested[proc] = FALSE;   /* už odcházím */
}
\end{verbatim}
 
\emph{Instrukce TSL} (spin-lock) - je nutné aby ji podporoval HW (všechny současné procesory nějakou mají):
\begin{verbatim}
enter_region:
    tsl   R,lock           ; načti zámek do registru R a
                           ; nastav zámek na 1
    cmp   R,#0             ; byl zámek nulový?
    jnz   enter_region     ; byl-li nenulový, znova
    ret                    ; návrat k volajícímu - vstup do
                           ; kritické sekce
 leave_region:
    mov   lock,#0          ; ulož do zámku 0
    ret                    ; návrat k volajícímu
\end{verbatim}

\subsubsection*{Pasivní čekání}
\emph{Vlastnosti}: proces je ve stavu blokován, vhodné pro delší doby čekání, spotřebovává prostředky OS, pomalejší.

Postup používající Sleep/Wakeup (implementovány OS, atomické operace - sleep uspí volající proces, wakeup probudí udaný proces) nefunguje (viď Problém producent/konzument).

\textbf{Semafory}... Sémantika:
\begin{verbatim}
Down(Semaphore s) {
  wait until s > 0, then s := s-1;
  /* must be atomic once s > 0 is detected */
}
\end{verbatim}
(pokud je čítač $>$ 0, sníží čítač o 1 a pokračuje dál; pokud je čítač $=$ 0, operace DOWN se zablokuje a proces je přidán do fronty čekající na tomto semaforu)

\begin{verbatim}
Up(Semaphore s) {
  s := s+1;   /* must be atomic */
}
\end{verbatim}
(pokud je fronta neprázdná, vybere libovolný proces a ten probudí za DOWN; jinak zvětší čítač o 1)

\begin{verbatim}
Init(Semaphore s, Integer v) {
  s := v;
}
\end{verbatim}

Je možné \uv{používať} aj sémantiku, kde sa hodnota vždy zníži/zvýši o 1 (a je možné sa teda dostať do záporných hodnôt semafóru)... Špeciálny (binárny) typ semaforu, kde sú povolené len hodnoty 0 a 1 (v Up sa miesto $s:=s+1$ volá $s:=1$) sa nazýva \emph{mutex} a používa sa na riadenie prístupu k jednej premennej.

\textbf{Monitory}
\par Implementovány překladačem, lze si představit jako třídu C++ (všechny proměnné privátní, funkce mohou být i veřejné), vzájemné vyloučení v jedné instanci (zajištěno synchronizací na vstupu a výstupu do/z veřejných funkcí, synchronizace implementována blokovacím primitivem OS). ???TODO

\textbf{Zprávy}
\par Operace SEND a RECEIVE, zablokování odesílatele/příjemce, adresace proces/mailbox, rendez-vous...

\textbf{RWL - read-write lock}, \textbf{bariéry}...

Ekvivalence primitiv - pomocí jednoho blokovacího primitiva lze implementovat jiné blokovací primitivum.

Rozdíly mezi platformami: Windows - jednotné funkce pro pasivní čekání, čekání na více primitiv, timeouty. Unix - OS implementuje semafor, knihovna pthread.

\subsubsection*{Klasické synchronizační problémy}
\textbf{Problém producent/konzument}
\par Producent vyrába predmety, konzument ich spotrebúva. Medzi nimi je sklad pevnej veľkosti (N). Konzument nemá čo spotrebúvať ak je sklad prázdny; producent prestane vyrábať, ak je sklad plný. 

\begin{verbatim}
int N = 100;
int count = 0;
void producer(void) {
    int item;
    while(TRUE) {
        produce_item(&item);
        if(count==N) sleep ();
        enter_item(item);
        count++;
        if(count == 1) wake(consumer);
    }
}
void consumer(void) {
    int item;
    while(TRUE) {
        if(count==0) sleep ();
        remove_item(&item);
        count--;
        if(count==N-1)
            wake(producer);
        consume_item(&item);
    }
}
\end{verbatim}

\begin{penumerate}
	\item Buffer je prázdny, a konzument práve prečítal count, aby zistil, či je rovný nule
	\item Preplánovanie na producenta
	\item Producent vytvorí item a zvýši count
	\item Producent zistí, či je count rovný jednej. Zistí že áno, čo znamená že konzument bol predtým zablokovaný (pretože muselo byť 0), a zavolá wakeup
	\item Teraz môže dôjsť k zablokovaniu: konzument sa uspí, pretože si myslí, že nemá čo zobrať; producent bude chvíľu produkovať a dôjde "preplneniu" $\Rightarrow$ uspí sa; spí producent aj konzument :o) 
\end{penumerate}

\textbf{Problém obědvajících filosofů}
\par Pět filosofů sedí okolo kulatého stolu. Každý filosof má před sebou talíř špaget a jednu vidličku. Špagety jsou bohužel slizké a je třeba je jíst dvěma vidličkami. Život filosofa sestává z období jídla a období přemýšlení. Když dostane hlad, pokusí se vzít dvě vidličky, když se mu to podaří, nají se a vidličky odloží.

\textbf{Problém ospalého holiče}
\par Holič má ve své oficíně křeslo na holení zákazníka a pevný počet sedaček pro čekající zákazníky. Pokud v oficíně nikdo není, holič se posadí a spí. Pokud přijde první zákazník a holič spí, probudí se a posadí si zákazníka do křesla. Pokud přijde zákazník a holič už střihá a je volné místo v čekárně, posadí se, jinak odejde.
