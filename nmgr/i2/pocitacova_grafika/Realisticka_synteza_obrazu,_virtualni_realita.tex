\newpage
\section{Realistická syntéza obrazu, virtuální realita}
\begin{pozadavky}
\begin{pitemize}
\item Metody reprezentace 3D scén, klasické zobrazovací algoritmy, výpočet viditelnosti, výpočet vržených stínů, modely osvětlení a stínovací algoritmy, rekurzivní sledování paprsku, textury, anti-aliasing, urychlovací metody pro ray-tracing, princip radiačních metod, výpočet konfiguračních faktorů, řešení radiační soustavy rovnic, hierarchické přístupy v radiačních metodách, fyzikální model šíření světla - zobrazovací rovnice, Monte-Carlo přístupy ve výpočtu osvětlení, hybridní zobrazovací metody, přímé metody ve vizualizaci objemových dat, generování izoploch, schéma grafického akcelerátoru, předávání dat do GPU, textury v GPU, programování GPU, základy jazyka Cg, pokročilé techniky práce s GPU, SW a HW prostředky pro virtuální realitu, vlastnosti jazyka VRML, struktura scény, typy uzlů (datové typy, trikové uzly), tvorba statické scény VRML, dynamické a interaktivní scény VRML, práce se skripty, rozhraní EAI, víceuživatelská virtuální realita.
\end{pitemize}
\end{pozadavky}