\newpage
\section{Základy počítačového zpracování přirozeného jazyka}
\begin{pozadavky}
\begin{pitemize}
\item Základy obecné lingvistiky (základní lingvistické pojmy a koncepty, strukturní lingvistika, typologie jazyků, funkce a forma).
\item Systém rovin popisu jazyka (fonetika, fonologie, morfologie, syntax povrchová/hloubková, sémantika, pragmatika).
\item Závislostní syntax, formální definice a vlastnosti závislostních stromů (závislosti, koordinace, projektivita).
\item Chomského hierarchie jazyků, bezkontextové jazyky, frázové gramatiky pro přirozený jazyk.
\item Návrh a vyhodnocení lingvistických experimentů, evaluační metriky (precision, recall, f-measure, statistická významnost a další).
\item Základní stochastické modely (generativní, diskriminativní; model zdrojového kanálu; HMM).
\item Jazykové modelování, základní metody trénování stochastických modelů (maximální věrohodnost, EM).
\item Základní algoritmy (Trellis, Viterbi, Baum-Welch).
\end{pitemize}
\end{pozadavky}