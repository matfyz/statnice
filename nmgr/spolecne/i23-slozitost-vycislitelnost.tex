\newpage
\section{Složitost}
\begin{pozadavky}
\begin{pitemize}
\item Metody tvorby algoritmů: rozděl a panuj, dynamické programování, hladový algoritmus.
\item Dolní odhady pro složitost třídění (rozhodovací stromy). 
\item Amortizovaná složitost. 
\item Úplné problémy pro třídu NP, Cook-Levinova věta.
\item Pseudopolynomiální algoritmy, silná NP-úplnost.
\item Aproximační algoritmy a schémata. 
\item Algoritmicky vyčíslitelné funkce, jejich vlastnosti, ekvivalence jejich různých matematických definic. 
\item Částečně rekurzivní funkce. 
\item Rekurzivní a rekurzivně spočetné množiny a jejich vlastnosti. 
\item Algoritmicky nerozhodnutelné problémy (halting problem). 
\item Věty o rekurzi a jejich aplikace: příklady, Riceova věta.
\end{pitemize}
\end{pozadavky}
\subsection{Metody tvorby algoritmů: rozděl a panuj, dynamické programování, hladový algoritmus}
\subsection{Dolní odhady pro uspořádání (rozhodovací stromy)}
\input{spolecne/slozitost/amortizovana_slozitost.tex}
\input{spolecne/slozitost/uplne_problemy_pro_tridu_np.tex}
\input{spolecne/slozitost/aproximacni_algoritmy_a_schemata.tex}
\subsection{Algoritmicky vyčíslitelné funkce, jejich vlastnosti, ekvivalence jejich různých matematických definic}
\input{spolecne/vycislitelnost/primitivne_a_castecne_rekurzivni_funkce.tex}
\input{spolecne/vycislitelnost/rekurzivni_a_rekurzivne_spocetne_mnoziny.tex}
\input{spolecne/vycislitelnost/algoritmicky_nerozhodnutelne_problemy.tex}
\input{spolecne/vycislitelnost/vety_o_rekurzi_a_jejich_aplikace.tex}