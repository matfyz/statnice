\newpage
\section{Složitost a vyčíslitelnost}
\begin{pozadavky}
\begin{pitemize}
\item Metody tvorby algoritmů: rozděl a panuj, dynamické programování, hladový algoritmus.
\item Dolní odhady pro složitost třídění (rozhodovací stromy). 
\item Amortizovaná složitost. 
\item Úplné problémy pro třídu NP, Cook-Levinova věta.
\item Pseudopolynomiální algoritmy, silná NP-úplnost.
\item Aproximační algoritmy a schémata. 
\item Algoritmicky vyčíslitelné funkce, jejich vlastnosti, ekvivalence jejich různých matematických definic. 
\item Částečně rekurzivní funkce. 
\item Rekurzivní a rekurzivně spočetné množiny a jejich vlastnosti. 
\item Algoritmicky nerozhodnutelné problémy (halting problem). 
\item Věty o rekurzi a jejich aplikace: příklady, Riceova věta.
\end{pitemize}
\end{pozadavky}
\input{spolecne/slozitost/Metody_tvorby_algoritmu.tex}
\input{spolecne/slozitost/Dolni_odhady_pro_usporadani_(rozhodovaci_stromy).tex}
\input{spolecne/slozitost/Amortizovana_slozitost.tex}
\subsection{Úplné problémy pro třídy NP, PSPACE, polynomiální hierarchie, pseudopolynomiální algoritmy}
\subsection{Aproximační algoritmy a schémata}
\input{spolecne/vycislitelnost/Algoritmicky_vycislitelne_funkce,_jejich_vlastnosti,_ekvivalence_jejich_ruznych_matematickych_definic.tex}
\subsection{Primitivně a částečně rekurzivní funkce}
\subsection{Rekurzivní a rekurzivně spočetné množiny a jejich vlastnosti}
\input{spolecne/vycislitelnost/Algoritmicky_nerozhodnutelne_problemy.tex}
\subsection{Věty o rekurzi a jejich aplikace}