\newpage
\section{Neprocedurální programování}
\begin{pozadavky}
\begin{pitemize}
\item Odlišnost procedurálního a neprocedurálního způsobu programování.
\item Principy funkcionálního a logického programování.
\item Lambda kalkulus, syntax, volné a vázané proměnné a principy redukce.
\item Churchova a Rosserova vlastnost a konsistence kalkulu.
\item Věty o pevném bodu.
\item Normální tvar objektů.
\item Typovaný lambda kalkul.
\item Curryho a Churchovy systémy typování.
\item Základní charakteristiky funkcionálních jazyků.
\item Hornova logika, Hornovy klausule.
\item Substituce, unifikace a jejich vlastnosti.
\item SLD-resoluce a logické programy.
\item Korektnost a úplnost SLD-resoluce.
\item Negace definovaná neúspěchem, obecné logické programy.
\item Čistý Prolog jako podmnožina Prologu.
\item Postačující podmínky ukončení výpočtu.
\item Unifikace bez kontroly výskytu proměnných.
\item Implementace Prologu.
\item Programování s omezujícími podmínkami: inferenční a prohledávací algoritmy splňování podmínek.
\end{pitemize}
\end{pozadavky}